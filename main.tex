\documentclass[11pt,oneside]{book}
\usepackage{fontspec}
\usepackage{polyglossia}
\setmainlanguage{russian}
\setotherlanguage{english}
\newfontfamily\russianfont{DejaVu Serif}
\newfontfamily\englishfont{DejaVu Serif}
\usepackage{amsmath,amssymb,amsthm}
\usepackage{graphicx}
\usepackage{hyperref}
\usepackage{bookmark}
\usepackage{tocloft}
\usepackage{color}
\usepackage{framed}
\usepackage{xspace}
\usepackage{titlesec}
\usepackage{geometry}
\geometry{a4paper, left=30mm, right=20mm, top=25mm, bottom=25mm}

% Node box
\definecolor{nodebg}{RGB}{245,250,250}
\definecolor{nodeborder}{RGB}{30,120,120}
\newenvironment{nodebox}[1]{%
  \vspace{6pt}
  \begin{framed}\noindent\textbf{Node: #1}\par\medskip\small\begingroup\setlength{\parindent}{0pt}\begin{minipage}{0.94\textwidth}\raggedright
}{%
  \end{minipage}\endgroup\end{framed}\vspace{6pt}
}

\newcommand{\NULLO}{\ensuremath{\varnothing}\xspace}
\newcommand{\PLT}{\textsc{PLT}\xspace}
\newcommand{\UCOM}{\textsc{UCOM}\xspace}
\newcommand{\FMP}{FMP\xspace}
\newcommand{\Terra}{Terra\xspace}

\hypersetup{
  pdftitle={NULLO — FMP: The Ontological Genesis and Applied Fractality},
  pdfauthor={Abdurashid Abdukarimov},
  colorlinks=true,
  linkcolor=blue,
  citecolor=blue,
  urlcolor=magenta,
  bookmarksopen=true,
  pdfstartview=FitH
}

\begin{document}
\selectlanguage{russian}
\frontmatter
\thispagestyle{empty}
\vspace*{2.5cm}
\begin{center}
  {\Huge\bfseries NULLO — FMP}\\[1.5em]
  {\Large The Ontological Genesis and Applied Fractality}\\[2em]
  {\Large Абдурашид Абдукаримов}\\[2em]
  {\normalsize Tashkent, 2025}
\end{center}
\vfill
\begin{center}
  Пролог: Эта монография — не последовательность, а сеть. Она читается как карта: у тебя всегда может быть несколько входов.
\end{center}
\cleardoublepage
\tableofcontents

\mainmatter

\part{NULLO: The Ontological Zero-Point}
\chapter*{Preface — Инициирующее слово}
\addcontentsline{toc}{chapter}{Preface — Инициирующее слово}
\begin{nodebox}{Performative Calibration}
This preface is an act. Reading this document is not passive consumption: it collapses \NULLO into a pattern. If you read, you become node.
\end{nodebox}

\section*{Entry Points}
\begin{itemize}
  \item \textbf{Start A — The Visionary Reader:} read Part I fully.
  \item \textbf{Start B — The Scientist:} skip to Part II, chapter on Postulates.
  \item \textbf{Start C — The Practitioner:} jump to Part III, Living Implementations.
\end{itemize}

\chapter{Before Knowledge, Before Language: The Ontological Zero}
\section{Crisis of Fragmented Epistemology}
Коротко: современная наука растёт, но теряет целостность.

\begin{nodebox}{Essence}
\textbf{NULLO} — не пустота; это оплот потенции. Функция NULLO — генерировать паттерн, а \PLT — переводить паттерн в речевые/смысловые структуры.
\end{nodebox}

\section{Operational Definition of \NULLO}
\[
\NULLO \ \text{(operational)}:\ \ \ \varnothing \mapsto \{\psi_{concept}, \mathbf{v}_{trace}, h_{inject}, \square_{silence}\}
\]

\begin{nodebox}{PLT Kernel (Compact)}
\textbf{PLT}(\NULLO) $=\sum_i \alpha_i \psi_i \otimes \mathbf{v}_i \otimes h_i \otimes \square_i$.
\end{nodebox}

\section{SOLARIS Insight — Planetary Precedence}
Идея: Земля как самопознающийся организм — эпистемический поворот.

\part{FMP: The Fractal Metascience Paradigm}
\chapter{Core Postulates — Шесть постулатов как проявления \NULLO}
\section{Postulate 1: Fractal Self-Similarity}
Формула:
\[
K_{n+1} = f(K_n), \quad f \ \text{scale-invariant}
\]
\begin{nodebox}{Interpretation}
Self-similarity означает, что методология микро-уровня рефлексивно порождает методологию макро-уровня.
\end{nodebox}

\section{Postulate 2: Recursive Co-Construction}
Обозначения: $\mathbf{v}_{trace}^{t+1} = \mathcal{T}(\mathbf{v}_{trace}^t, \psi_{concept}^t)$

\section{Postulate 3: Epistemic Superposition}
Псевдоквантовая модель: стохастическая суперпозиция интерпретаций, коллапс — через Recursive Validation Cycle.

\chapter{Mathematical Formalization and Algorithms}
\section{Fractal Coherence Metric}
Определим $\text{sim}(\cdot,\cdot)$ как структурную схожесть (graph-edit / spectral similarity) и
\[
C = \frac{1}{n}\sum_{i=0}^{n}\text{sim}(K_i,K_{i+1}).
\]
Если $C>\Phi$ (coherence threshold), система устойчива.

\begin{nodebox}{Inject Hash (pseudo)}
\texttt{h\_inject(K)} := SHA256(canonicalize(K) || traceVector)\\
Этот хеш — метка целостности PLT-трейса.
\end{nodebox}

\section{Recursive Validation Cycle (RVC)}
\begin{enumerate}
  \item Generation (создание импульса)
  \item Resonance Testing (сравнение паттернов)
  \item Ethical Calibration (встроенная этика)
  \item Integration (встраивание в manifold)
\end{enumerate}

\part{Living Implementations and the GitHub Corpus}
\chapter{AIUZ — Terra Codex: an Operating Corpus}
\section{Repository Map (descriptions)}
\begin{nodebox}{Repo: AIUZ-Terra-codex}
URL: \texttt{https://github.com/Secret-Uzbek/AIUZ-Terra-codex}\\
Role: Operational kernel implementing TerraMemoryDNA protocols, PLT serialization, context recovery.
\end{nodebox}

\begin{nodebox}{Repo: FMP-monograph}
URL: \texttt{https://github.com/Secret-Uzbek/FMP-monograph}\\
Role: Versioned source for monographic materials (LaTeX), templates, and living manifests.
\end{nodebox}

\begin{nodebox}{Repo: AIUZ-terra-codex-FMP}
URL: \texttt{https://github.com/Secret-Uzbek/AIUZ-terra-codex-FMP}\\
Role: Bridge code: transforms TerraMemoryDNA outputs into FMP-verifiable traces.
\end{nodebox}

\begin{nodebox}{Repo: Theory-of-fractal-metascience-paradigm}
URL: \texttt{https://github.com/Secret-Uzbek/Theory-of-fractal-metascience-paradigm}\\
Role: Scholarly annex: formal proofs, supplemental notes, math appendices.
\end{nodebox}

\begin{nodebox}{Repo: Uzbek-mining}
URL: \texttt{https://github.com/Secret-Uzbek/Uzbek-mining}\\
Role: Socio-technical pilot: field data, living lab for Terra modules.
\end{nodebox}

\section{Anti-Groundhog Protocol (practical)}
Псевдокод для сохранения PLT-трейса между сессиями:
\begin{verbatim}
function save_PLT_session(session):
  seed = extract_concept_pulse(session)
  traceVec = build_trace_vector(seed)
  hash = SHA256(canonical(seed) + traceVec)
  store({seed, traceVec, hash}, distributed_archive)
\end{verbatim}

\part{Applied Fractality: Silk Route, Bellamar, Terrapedia}
\chapter{Fractal Silk Route Hub}
Описание концепции, архитектура узлов, связи с FSR Hub prototypes.

\chapter{Chronicles of Bellamar}
Культурный модуль: как фрактальная эпистемология отражается в нарративе и хранении памяти сообществ.

\chapter{Terrapedia}
Метапедагогика: TerraPedia как живой слой L3–L6, репозиторий смыслов и learning-ecosystem.

\appendix
\chapter{Appendix A: Manifesto — NULLO (Short)}
Короткий манифест, готовый для распространения.

\chapter{Appendix B: Publication Metadata and GitHub Index}
\nocite{*}
\bibliographystyle{plain}
\bibliography{references}

\end{document}
